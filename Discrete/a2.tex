\documentclass[journal,12pt,twocolumn]{IEEEtran}
\usepackage{cite}
\usepackage{amsmath,amssymb,amsfonts,amsthm,mathtools}
\usepackage{algorithmic}
\usepackage{graphicx}
\parindent 0px
\bibliographystyle{IEEEtran}
\title{NCERT DISCRETE 11.9.2 Q10}
\author{EE23BTECH11052 - Abhilash Rapolu }
\begin{document}
\maketitle
\newpage
\textbf{Question 11.9.2.10}:If the sum of first $p$ terms of an A.P. is equal to the sum of the first $q$ terms, then
find the sum of the first $(p + q)$ terms.\\
\ Solution:
\begin{table}[htbp]
\centering

\begin{document}
\begin{tabular}{|c|c|c|}
\hline
\textbf{Parameter}&\textbf{Value}\\
   \hline
    ${f_{max}}$  & $12 MhZ$ \\
   \hline
   $f_{min}$ & $7.5 MhZ$ \\
   \hline
\end{tabular}
\end{document}

\caption{Given parameters list}
\end{table}
Now let's find the z transform of the $x(n)$ using the linearity property.\\
\begin{align}
X(z)&=\frac{a_0}{1-z^{-1}}+d\frac{z^{-1}}{(1-z^{-1})^2}\\
y(n) &= x(n)*u(n)
\end{align}
Now apply z transform on both sides\\
\begin{align}
Y(z)&=X(z)U(z)\\
Y(z)&=\frac{a_0}{(1-z^{-1})^2}+d\frac{z^{-1}}{(1-z^{-1})^3}
\end{align}
by comparison of the above equations:\\
\eqref{ZPT}\\
the inverse z transform:\\
\begin{align}
y(n)&=a_0(n)+\frac{d}{2}(n)(n-1)
\end{align}
as we considered n=0 as our first term, we have to replace n by (n+1)\\
Sum of first n terms is given as:\\
\begin{align}
y(n)&=a_0(n+1)+\frac{d}{2}(n+1)(n)
\end{align}
given in question y(p-1)=y(q-1)\\
\begin{align}
a_0(p)+\frac{d}{2}(p-1)(p)&=a_0(q)+\frac{d}{2}(q-1)(q)\\
d&=(-)\frac{2a_0}{p+q-1}
\end{align}
now for first p+q terms:\\
\begin{align}
y(p+q-1)&=a_0(p+q)+\frac{d}{2}(p+q-1)(p+q)
\end{align}
substitue d in this\\
\begin{align}
y(p+q-1)&=a_0(p+q)-\frac{a_0}{p+q-1}(p+q-1)(p+q)\\
y(p+q-1)&=a_0(p+q)-a_0(p+q)\\
y(p+q-1)&=0.
\end{align}

\end{document}

