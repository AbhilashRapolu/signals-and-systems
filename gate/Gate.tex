% \iffalse
\let\negmedspace\undefined
\let\negthickspace\undefined
\documentclass[journal,12pt,twocolumn]{IEEEtran}
\usepackage{cite}
\usepackage{amsmath,amssymb,amsfonts,amsthm}
\usepackage{algorithmic}
\parindent 0px
\bibliographystyle{IEEEtran}
\vspace{3cm}
\title{GATE 2023-EE Q49}
\author{EE23BTECH11052 - Abhilash Rapolu $^{}$% <-this % stops a space
}
\begin{document}
\maketitle
\newpage
\bigskip
\ QUESTION:The period of the discrete-time signal x[n] described by the equation below is N =\ (Round off to the nearest integer).
$$x[n] = 1 + 3\sin\left(\frac{15\pi}{8}n + \frac{3\pi}{4}\right) - 5\sin\left(\frac{\pi}{3}n - \frac{\pi}{4}\right)$$
\ SOLUTION:  

The signal can be expressed as the sum of two sinusoids:

 Sinusoid 1: Frequency $$(f_1) = \frac{15\pi}{8\pi} = \frac{15}{16}$$
 Sinusoid 2: Frequency $$(f_2) = \frac{\pi}{6\pi} = \frac{1}{6}$$

Therefore, the frequency components of $x[n]$ are:

$$f_1 = \frac{15}{16} \quad \text{and} \quad f_2 = \frac{1}{6}$$
 $$T_i = \frac{1}{f_i}$$ 
\\The time period must be an integer for a discrete time signal.
\\Therefore, we need to find the smallest integer $N$ that is a multiple of both $T_1$ and $T_2$:

 $$T_1 = \frac{1}{f_1} = \frac{16}{15}$$ (not an integer)
 $$T_2 = \frac{1}{f_2} = 6$$ (an integer)

Since $T_1$ is not an integer. However, $T_2$ is an integer,and it is a factor of $\frac{16}{15}$ (15 divides into 16 exactly once). 
\\Therefore, the smallest $N$ that satisfies the periodicity condition is:

$$N = \text{LCM}(T_1, T_2) = \text{LCM}(16, 6) = 48$$
\\The frequency components of the signal are $f_1 = \frac{15}{16}$ and $f_2 = \frac{1}{6}$. The time period of the signal is $$N =48$$.
\end{document}





