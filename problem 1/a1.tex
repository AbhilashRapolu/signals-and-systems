% \iffalse
\let\negmedspace\undefined
\let\negthickspace\undefined
\documentclass[journal,12pt,twocolumn]{IEEEtran}
\usepackage{cite}
\usepackage{amsmath,amssymb,amsfonts,amsthm}
\begin{document}
\bibliographystyle{IEEEtran}
\vspace{3cm}
\title{NCERT 11.15. Q10}
\author{EE23BTECH11052 - Abhilash Rapolu $^{*}$% <-this % stops a space
}
\maketitle
\newpage
\bigskip
\renewcommand{\thetable}{\arabic{table}}
\bibliographystyle{IEEEtran}
\parindent 0px
\Question A radio can tune in to any station in the 7.5 MHz to 12 MHz band.
 What is the corresponding wavelength band? 
\Solution  
\begin{table}[htbp] \small
\centering

\begin{document}
\begin{tabular}{|c|c|c|}
\hline
\textbf{Parameter}&\textbf{Value}\\
   \hline
    ${f_{max}}$  & $12 MhZ$ \\
   \hline
   $f_{min}$ & $7.5 MhZ$ \\
   \hline
\end{tabular}
\end{document}

\caption{Given \, parameters list}\end{table}
The wavelength ($\lambda$) of a radio wave is inversely proportional to its frequency (f).
\bgroup \obeylines
$$\lambda=\frac{c}{f}$$
$$\lambda_{max}=\frac{c}{f_{min}}$$
\egroup
\begin{align}
\lambda_{max}=\frac{(3\times10^{8})}{(7.5\times10^{6})}=40
\end{align}
For 12MHz:
$$\lambda_{min}=\frac{c}{f_{max}}$$
\begin{align}
\lambda_{min}=\frac{(3\times10^{8})}{(12\times10^{6})}=25
\end{align}
Therefore, the radio can tune in to wavelengths ranging from 25 meters to 40 meters.
\end{document}
