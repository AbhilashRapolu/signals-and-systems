% \iffalse
\let\negmedspace\undefined
\let\negthickspace\undefined
\documentclass[journal,12pt,twocolumn]{IEEEtran}
\usepackage{cite}
\usepackage{amsmath,amssymb,amsfonts,amsthm}

\begin{document}

\bibliographystyle{IEEEtran}

\vspace{3cm}
\title{NCERT 11.15. Q10}
\author{EE23BTECH11052 - Abhilash Rapolu $^{*}$% <-this % stops a space
}
\maketitle

\newpage
\bigskip
\renewcommand{\thetable}{\arabic{table}}
\bibliographystyle{IEEEtran}
\parindent 0px

\ Question
A radio can tune in to any station in the 7.5 MHz to 12 MHz band. What is the corresponding wavelength band?

\ Solution
The wavelength ($\lambda$) of a radio wave is inversely proportional to its frequency (f).

\begin{table}[htbp] \small
\centering
\begin{tabular}{|l|l|c|}
\hline
\textbf{Parameter} & \textbf{Description} & \textbf{Value} \\
\hline
$f_{\text{max}}$ & Maximum Frequency & 12 MHz \\
\hline
$f_{\text{min}}$ & Minimum Frequency & 7.5 MHz \\
\hline
\end{tabular}


\caption{Given \, parameters list}
\end{table}

For 7.5 MHz:
\begin{align}
\lambda_{max} &= \frac{c}{f_{min}} \\
&= \frac{(3\times10^{8})}{(7.5\times10^{6})} \\
&= 40 \text{ meters}
\end{align}

For 12 MHz:
\begin{align}
\lambda_{min} &= \frac{c}{f_{max}} \\
&= \frac{(3\times10^{8})}{(12\times10^{6})} \\
&= 25 \text{ meters}
\end{align}

Therefore, the corresponding wavelength band is from 25 meters to 40 meters.

\end{document}

