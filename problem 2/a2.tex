% \iffalse
\let\negmedspace\undefined
\let\negthickspace\undefined
\documentclass[journal,12pt,twocolumn]{IEEEtran}
\usepackage{cite}
\usepackage{amsmath,amssymb,amsfonts,amsthm}
\usepackage{algorithmic}
\parindent 0px
\bibliographystyle{IEEEtran}
\vspace{3cm}
\title{NCERT 11.9. Q2}
\author{EE23BTECH11052 - Abhilash Rapolu $^{*}$% <-this % stops a space
}
\begin{document}
\maketitle
\newpage
\bigskip
\ Question: If the sum of first $p$ terms of an A.P. is equal to the sum of the first $q$ terms, then
 find the sum of the first $(p + q)$ terms.
\ Solution:  
\\The sum of first $p$ terms of an arithmetic progression (A.P) is given by
$$s_p =\frac{q}{2}[2a+(q)d]$$
If $s_p$=$s_q$,then:
$$\frac{p}{2}[2a+(p)d]=\frac{q}{2}[2a+(q)d]$$
simplifying the equation we get:
\begin{align}
p*(2a+pd)=q*(2a+qd)
2ap+(p^2)*d=2aq+(q^2)*d
2a(p-q)+(p-q)(p+q)*d=0
(p-q)[2a+(p+q)*d]=0
\end{align} 
since $p$ and $q$ are not equal.We can eliminate the term $(p-q)$
Now the equation becomes :
$$2a+(p+q)*d=0\longrightarrow{1}$$
Now to find the sum of the first $p+q$ terms $S_{p+q}$,you can use the formula:
$$S_{p+q}=\frac{p+q}{2}[2a+(p+q)*d]$$
As we have seen in the equation $1$ : $2a+(p+q)*d=0$ is $0$.
 Therefore $S_{p+q}$ is $0$.
\end{document}

